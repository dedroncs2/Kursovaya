\documentclass[12pt]{article}
\usepackage[]{amsmath,amsthm,amssymb}
\usepackage[T1,T2A]{fontenc}
\usepackage[utf8]{inputenc}
\usepackage[english,russian]{babel}
\usepackage[]{graphicx}
\usepackage[]{subcaption}

\begin{document}
    \section{Постановка задачи}
    \subsection{Цель}
    Цель курсовой работы - упростить процессы сбора и обработки данных с устройств "IoT".
    \subsection{Исходные информация}
    Исходными данными являются результаты анализа подобных систем.
    \subsection{Априорные модельные представления}
    Информационная система должна позволять пользователю:
    \begin{itemize}
        \item создать свою учетную запись;
        \item добавлять свои устройства;
        \item получать данные со своих устройств IoT;
        \item хранить/удалять переданные данные;
        \item представлять/отображать свои данные при помощи графиков и полей;
        \item настроить оповещения о событиях, связанных с полученными данными;
        \item иметь обратную связь со своими устройствами;
    \end{itemize}
    \subsection{Ожидаемый результат}
    Проект информационной системы с полной реализацией, позволяющий обрабатывать и представлять информацию
    и удовлетворять априорным модельным представлениям.
    \subsection{Критерий оценки результата}
    Пользователь может подключить любое IoT устройство при помощи API и сохранять и/или отображать на графиках.
    \subsection{Средства проектирования и реализации}
    Для проектирования информационной системы были выбраны средства:
    \begin{itemize}
        \item Visual Paradigm Community Edition;
        \item ERwin Process modeller;
    \end{itemize}
    \subsection{Требования к информационной системе (ИС)}
    Функциональные требования:
    \begin{itemize}
        \item ИС идентифицирует IoT устройства; 
        \item ИС проверяет тип полученных данных;
        \item ИС содержит базу данных пользователей и их данных;
        \item ИС предоставляет возможность отображения полученной информации различными средствами (поля, графики);
        \item ИС обладает средствами по отправке оповещений связанных с событиями, описанными пользователем;
        \item ИС может связываться с, и отправлять команды на соответствующее IoT устройство;
    \end{itemize}
    Внефункциональные требования:
    \begin{itemize}
        \item Бэк-энд реализация на ASP.NET Core 2.0 сервере и языке C\#;
        \item Минималистичный, быстрый и отзывчивый фронтэнд без использования больших популярных
         фреймворков;
        \item используется REST архитектура;
    \end{itemize}
\end{document}