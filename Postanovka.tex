\documentclass[12pt]{report}
\usepackage[]{amsmath,amsthm,amssymb}
\usepackage[T1,T2A]{fontenc}
\usepackage[utf8]{inputenc}
\usepackage[english,russian]{babel}
\usepackage[]{graphicx}
\usepackage[]{subcaption}

\begin{document}
    \section{Постановка задачи}
    \subsection{Цель}
    Цель курсовой работы - упростить процессы сбора и обработки данных с устройств "IoT".
    \subsection{Исходные информация}
    Исходными данными являются результаты анализа подобных систем.
    \subsection{Априорные модельные представления}
    Информационная система должна позволять пользователю:
    \begin{itemize}
        \item создать свою учетную запись
        \item добавить свои устройства 
        \item хранить/удалять переданные данные
        \item представлять свои данные при помощи графиков и полей
        \item управлять своими устройствами через API
    \end{itemize}
    \subsection{Ожидаемый результат}
    Проект информационной системы с полной реализацией, позволяющий обрабатывать и представлять информацию
    и удовлетворять априорным модельным представлениям.
    \subsection{Критерий оценки результата}
    Пользователь может подключить любое IoT устройство при помощи API и сохранять и/или отображать на графиках.
    \subsection{Средства проектирования и реализации}
    Для проектирования информационной системы были выбраны средства:
    \begin{itemize}
        \item Dia diagram editor
        \item ???
    \end{itemize}
    \subsection{Требования к информационной системе (ИС)}
    Функциональные требования:
    \begin{itemize}
        \item ИС проверяет тип полученных данных
        \item ИС имеет базу данных пользователей и их данных
        \item ИС может по разному представлять полученные данные
        \item ИС должна иметь двухстороннюю связь с устройствами
    \end{itemize}
    Внефункциональные требования:
    \begin{itemize}
        \item Бэк-энд реализация на ASP.NET сервере и языке C\#
    \end{itemize}
\end{document}